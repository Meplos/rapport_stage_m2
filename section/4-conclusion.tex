\section{Conclusion}

Après avoir participé à plusieurs petits projets et 2 gros (développer ci-dessus), j'ai appris l'importance de la communication entre les équipes. J'ai été amener à réaliser plusieurs petit script pour aider les chefs de projets dans le studio digital, lors de ces missions, j'ai fais tout mon possible pour expliquer le fonctionnement du code pour qu'il puisse si nécessaire le réutiliser dans d'autre projet pour des situations similaires. En plus de ces missions, j'ai essayé le plus de partager avec mon chef de projet, afin de lui faire pars de mes avancés, mes problématiques et mes idées. La reprises d'un projet existant m'a également beaucoup appris sur le passage de connaisances entre développeur. En effet l'une des principales difficulté lors de ce projet à était la reprinse d'un code qui sur certains aspect mal documenter. 

Du côté plus technique, j'ai beaucoup appris sur le fonctionnement général de javascript, en plus d'avoir approfondis mes compétences technique en VueJs tout en améliorant mes compétences en veille technologique. Pour ce qui ai du backend, j'ai essayé au plus possible de documenter mes requête afin de faciliter leur compréhension mais également d'integrer des tests à l'aide de postman. Ces tests me permetter facilement de savoir si une fonctionalité existante a été impacter par les modifications du code. 

Pour conclure, ce stage m'a donc permis d'approfondir mes compétences sociales et technique. Il m'a conforter dans l'idée de faire du développement web mon métier. De plus le déroulement de mon stage s'étant bien passer j'ai donc choisie de continuer à travailler au poste de développeur fullstack dans l'entreprise Cirrusware. 