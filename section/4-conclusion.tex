\section{Conclusion}

Après avoir participé à plusieurs petits projets et 2 gros (développés ci-dessus), j'ai appris l'importance de la communication entre les équipes. J'ai été amené à réaliser plusieurs petits scripts pour aider les chefs de projets dans le studio digital. Lors de ces missions, j'ai fait tout mon possible pour expliquer le fonctionnement du code pour qu'ils puissent si nécessaire le réutiliser dans d'autres projets pour des situations similaires. En plus de ces missions, j'ai essayé le plus possible  de partager avec mon chef de projet, afin de lui faire part de mes avancés, mes problématiques et mes idées. La reprise d'un projet existant m'a également beaucoup appris sur le passage de connaisances entre développeurs. En effet l'une des principales difficultés lors de ce projet a été la reprise d'un code qui sur certains aspects mal documenté. 

Du côté plus technique, j'ai beaucoup appris sur le fonctionnement général de javascript, en plus d'avoir approfondi mes compétences technique en VueJs tout en améliorant mes compétences en veille technologique. Pour ce qui est du backend, j'ai essayé au plus possible de documenter mes requêtes afin de faciliter leur compréhension mais également d'intégrer des tests à l'aide de postman. Ces tests m'ont permis de facilement savoir si une fonctionalité existante a été impactée par les modifications du code. 

Pour conclure, ce stage m'a donc permis d'approfondir mes compétences sociales et techniques. Il m'a conforté dans l'idée de faire du développement web mon métier. De plus le déroulement de mon stage s'étant bien passé j'ai donc choisi de continuer à travailler au poste de développeur fullstack dans l'entreprise Cirrusware. 