\section{Outils et technologie utilisées}

//TODO phrase d'intro
A mon arrivé dans l'entreprise, j'ai du m'habitué a utilisé leur outils de, gestion de projet, de communication, mais aussi de developement. Lors de nos projets d'études, là ou on était une équipe de maximum 6 et on utilisé principalement des outils tels que Discord pour communiquer, VSCode pour developper, et GitHub Issue ou trello pour gérer le projet. Ici j'ai du m'habitué à utiliser des outils plus proffesionel et complet afin de pouvoir travailler sur des projets de plus grande taille. Voici la liste exhaustive de ces outils. 
\subsection{Outils}

\subsubsection{Monday}

Monday est une outils de gestion de projet simple qui permet de créer des tableaux et y ajouter des tâches.
Chaque tâche peut etre attribué à un quelqu'un, posséde un état (en cours, terminé, etc.) et une importance.
De plus chaque tâche posséde un espace "discussion" qui permet de faire état de l'avancement ou encore de noter certaine informations importantes. 

\begin{figure}[htbp]
    \center
    //TODO Screen tableaux Monday
    \caption{Tableau Monday}
\end{figure}

\subsubsection{Google Workspace}
Afin de faciliter la communication au seins des équipe l'entreprise utilise Google Workspace afin de produire de la documentation (Cahier des charges, etc...), GMail et Hangout afin d'échanger rapidement et Agenda pour planifier des rendez vous.
De plus, l'entreprise a des client se trouvant dans toute la France donc afin de faciliter les rendez vous projet l'applications Google Meet est utilisé comme application de visio conférence. 

\subsubsection{Git et GitKraken}
Git est un logiciel de gestion de version, il permet de créer des projets et de les versionner. Il permet également le partage du code ainsi que le développement sur plusieurs branche, afin de ne pas empiéter sur le travail des autre membres de l'équipe.

GitKraken lui est une interface graphique utilisant git. En effet git s'utilisant pricipalement par ligne de commande il est parfoit difficile de l'utiliser. GitKraken permet de faciliter l'utlisation de git et permet de tout faire en passant par l'interface (pull, merge, se déplacer entre les branche, etc..).

\begin{figure}[htbp]
    \center
    //TODO Screen GitKraken
\end{figure}

De plus il permet une total intégration des git flow. C'est a dire créer facilement un branche par nouvelle feature, et simplement merge avec la branche de développement lorsque celle ci est terminé.
Les git flow permette également une meilleure gestion des HotFix et des release mais je ne les ai pas utilisé lors de mon stage. 

\begin{figure}[htbp]
    \center
    //TODO Screen GitFlow
\end{figure}


\subsubsection{PHPStorm}

PHPStorm créer par JETBRAIN et l'un des IDE les plus complet pour dévellopement web.
A l'aide de ses nombreux pluggins il permet une total prise en charge des framework comme Symphony et VueJS. De plus il permet de facilement refactor du code  ou encore de créer de nouveaux snippet afin de faciliter les phase de dévellopement.








