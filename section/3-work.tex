\section{Missions réalisées}
Au cours de mon stage j'ai pu intervenir sur différents projets allant d'un simple script permettant de dynamiser une page web d'un client à la reprise d'un projet complet. Ici nous allons détailler les deux principales missions de mon stage à savoir le développement de widget en vueJS pour le studio Digital et la finalisation des XRM du \textit{Consortium du Jambon de Bayonne} et d'\textit{Interporc Franche-Compté}.

\subsection{Widget pour le studio Digital}
Lors de mon arrivé dans l'entreprise, celle ci cherchait a valorisé ses formation en devenant Organisme de formation certifié \textbf{QUALIOPI}. QUALIOPI est une certification qui garanti des formations avec: 
\begin{itemize}
    \item des bons résultats
    \item des adaptations au besoins
    \item des formateurs qualifiés
    \item une bonne prise en compte des retours face au formations
\end{itemize} 
Afin de correspondre à ces critères, nous avions besoin de construire un site web qui permettrait d'effectuer les questionnaires des différents modules de la formations, ainsi que le formulaire de satisfaction. Puis par la suite de pouvoir consulter les différentes statistiques des différents modules sous 3 vue différentes, formés, qui doit voir uniquement ses résultats, entreprise qui ne voit que les statistiques de ses employés formés, une visions globale qui permettra de voir toutes les statistiques des personnes formées.

La première étape réalisé afin de pouvoir réaliser ceci à était d'adapter l'enregistrement des formulaires du studio qui jusque là ne nous permet pas d'effectuer simplement des statistiques. 

\subsubsection{Serializations des données des formulaires}