\section{Contexte du stage}
\subsection{Cirrusware}
Cirrusware est une PME informatique française spécialisée dans le développement d'applications web ayant pour objectif de faciliter les communications numériques au sein des entreprises ainsi que de faciliter la gestion des relations entre clients et contacts. 

Fondée en 2013, Cirrusware a su convaincre de grands groupes français tels que \textit{RANDSTAD}, \textit{SAFRAN}, \textit{MAÏSADOUR}, \textit{OCEALIA}, \textit{CARREFOUR}, ou encore \textit{INTERCONTINENTAL - grand hôtel de Bordeaux}.


Aujourd'hui l'entreprise compte 13 employés dont 6 développeurs 5 chefs de projet et 2 commerciaux. 


Cirrusware développe 2 type principaux de produits: 
\begin{list}{}{}
    \item La plateforme Send-Up\cite{sendup}
    \item Un XRM (CRM).
\end{list}

\subsection{La plateforme Send Up}
\begin{figure}[!h]
\centering
    \includegraphics[width=0.4\textwidth]{/logo_su.png}

\end{figure}
Send-Up est le premier produit de Cirrusware et permet au client de faciliter leur communication marketing auprès de leurs différents contacts.
Les principales fonctionalités de la plateforme sont:
\begin{itemize}
    \item La création de campagne Emailing, SMS. 
    \item La création de Site/Landing Page via leur Studio digitale.
    \item L'automatisation de campagnes ou autre tâches répétives via l'automation.
    \item Stockage cloud 
\end{itemize}


En plus de campagne Emailing personnaliser en fonction des destinataires. La plateforme possède également un éditeur PDF afin de réaliser des campagnes papiers. 

Le Studio digitale est lui un éditeur WYSIWYG\footnote{What you see is what you get} de page web. Il permet en glissant simplement des blocs paramétrables afin d'atteindre le résultat souhaité. Il permet aussi d'accéder facilement au code source de la page afin d'ajouter des fonctionalités et comportements à la page pour les clients qui font une utilisation plus poussée de l'outil.

L'automation lui permet d'automatiser les tâches les plus répétives de la coommunication marketing, comme l'envoie d'une campagne toutes les semaines, ou encore l'envoie d'un mail à chaque inscription sur le site d'un client. 

En plus de fournir la plateforme. Cirrusware propose d'accompagner ses clients dans leurs différents projets. Allant de la création de campagnes efficaces à l'intégration de sites web. L'entreprise propose aussi des formations affin de permettre à leurs clients d'être automomes dans l'utilisation des fonctionalités les plus poussées de la plateforme.

\subsection{Le XRM}
Le XRM est un outil permettant d'optimiser la relations entre clients, contacts et fournisseurs. Il est totalement intégré à la platforme Send-Up.

Il permet: 
\begin{itemize}
    \item une gestion de portefeuille clients. 
    \item une gestion des actions (rendez-vous, notes, comptes-rendus, emails, sms)
    \item une gestion de diverses oportunitées. 
    \item utilsation Hors ligne
    \item une adaptation au besoin du client. 
\end{itemize}

Le XRM est un outil très modulable qui permet donc pour chaque nouveau client d'intégrer des fonctionalités uniques. Comme la visualisation de statistiques particulières, ou encore la création et la gestion d'audit\footnote{"L’audit est une expertise professionnelle effectuée par un agent compétent et indépendant aboutissant à un jugement par rapport à une norme sur les états financiers, le contrôle interne, l'organisation, la procédure, ou une opération quelconque d'une entité." - \textit{Wikipedia}\cite{audit}}. 

